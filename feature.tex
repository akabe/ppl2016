\begin{block}{特徴関数:単純型付け規則を弱めた規則}
  \vskip1ex
  \begin{tabular}{l@{~}l}
    例:
    & $\text{\textsc{T-WeakIf1}}\,
      \dfrac
      {~M : \alert{\tau'}\quad N : \tau\quad K : \tau~}
      {~\IF~M~\THEN~N~\ELSE~K : \tau~}$ \\[1em]
    & $\text{\textsc{T-WeakIf2}}\,
      \dfrac
      {~M : \BOOL\quad N : \tau\quad K : \tau~}
      {~\IF~M~\THEN~N~\ELSE~K : \alert{\tau'}~}$ \\[1em]
    & $\text{\textsc{T-WeakIf3}}\,
      \dfrac
      {~M : \BOOL\quad N : \tau\quad K : \alert{\tau'}~}
      {~\IF~M~\THEN~N~\ELSE~K : \tau~}$ \\[1em]
    & $\text{\textsc{T-WeakIf4}}\,
      \dfrac
      {~M : \BOOL\quad N : \alert{\tau'}\quad K : \tau~}
      {~\IF~M~\THEN~N~\ELSE~K : \tau~}$
  \end{tabular}
  \vskip1ex
  \begin{itemize}
  \item 型付け可能な式:\alert{弱めた規則のみ}で\alert{正しく推論}可能
    \begin{itemize}
    \item 点数最大化のため全ての規則を同時に適用し \textsc{T-If} と同じ規則になる
    \end{itemize}
  \item 型付け不可能な式:\alert{well-typed らしさ}を計算
    \begin{itemize}
    \item 型付け不可能な$\lambda$式でも可能な限り型を合わせる
    \item より多くの規則が成立すれば、点数が上がる
    \end{itemize}
  \end{itemize}
\end{block}